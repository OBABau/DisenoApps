\documentclass{article}
\usepackage{graphicx} % Required for inserting images

\title{Mobile Development Fundamentals}
\author{Brian Bautista}
\date{January 2024}

\begin{document}

\maketitle

\section{Introduction}
This document explores the fundamentals for mobile development.

\section{Simplicity}
The mobile version of a website should be as lightweight as possible. One way to achieve this is by omitting heavy graphics and superfluous content, which can hinder navigation on a small screen. Single-column layouts are more effective for mobile devices.

\section{Usability}
Sufficient space should be given for selecting links or interacting with the site. It is crucial to create a format that is easy to navigate. To facilitate navigation and make it user-friendly, information should be classified with logically ordered content and clearly labeled categories.

\section{Loading Time}
Mobile web users want to access content quickly and easily, and slow loading times will cause them to leave the page. Therefore, the simplicity of the website design is important. Complex elements such as large images, videos, redirects, etc., can significantly increase loading time, and most of these may not function the same on all mobile devices, potentially leading to blank spaces.

\section{Focused Content}
Understanding what information users are looking for on a mobile site and why they access a portal through a mobile device helps determine what content to publish, in what style, and length.

\section{Having a Solid Engine}
This will allow modifications as construction continues. A stable and problem-free engine will also allow updating and adding content to the site without encountering errors and crashes.

\section{Single Screen}
Even with smartphones that allow users to run multiple applications or keep several browser windows open simultaneously, results are displayed on a single screen set for interaction, suggesting that we need to focus on creating experiences on a single screen.

\section{Use of Established Device Standards}
It is advisable to use consistent user interface patterns that exist (for example: the back button in the top left), if we choose not to adhere to standards, we must ensure that we do so for specific and correct reasons.

\section{Limited Resources}
Despite technology for smartphones advancing every day, there is still limitation in the quality of connection, battery life, processor power, internal memory, among others.

\end{document}
