\documentclass{article}
\usepackage{geometry}
\usepackage{graphicx}
\usepackage{titlesec}
\usepackage{enumitem}

% Margin and spacing adjustments
\geometry{margin=1in}
\setlength{\parindent}{0pt}
\setlength{\parskip}{10pt}

% Title configuration
\titleformat{\section}[block]{\normalfont\Large\bfseries}{\thesection}{1em}{}
\titlespacing*{\section}{0pt}{\baselineskip}{\baselineskip}

% List configuration
\setlist[itemize]{left=0em,label=--}

\title{Design Patterns for Mobile Applications}
\author{Brian Bautista}
\date{January 2024}

\begin{document}
\maketitle

\section{Introduction}
Design patterns have always helped in creating manageable, testable, reusable, and optimized software. Generally, it helps in modularizing the software in such a way that each component is separated and handles a single responsibility. Additionally, it drastically enhances code readability, which plays a significant role in communicating the software code. Moreover, the software development process is accelerated dramatically with the already proven design paradigms. Mobile device developers are not far behind in reaping the benefits of following design patterns. Initially, mobile applications were too small to follow design or architectural patterns and therefore used to strictly adhere to the most basic ones. Nowadays, as mobile applications are becoming larger and almost reflecting (in terms of functionality) their desktop or web counterparts, they need to consider design patterns before they really go into development mode.

\section{MVC (Model View Controller)}
MVC stands for Model-View-Controller. This is what each of those components mean:

\textbf{Model:} The backend that holds all the data logic.

\textbf{View:} The frontend or graphical user interface (GUI).

\textbf{Controller:} The brain of the application that controls how data is displayed.

\textbf{Why should you use MVC?}

Three words: separation of concerns, or SoC for short.

The MVC pattern helps you to split frontend and backend code into separate components. This way, it's much easier to manage and make changes to either side without interfering with each other.

But this is easier said than done, especially when multiple developers need to update, modify, or debug a completed application simultaneously.

\section{MVP (Model View Presenter)}
MVP is based on the principle of separation of concerns and promotes more modular and maintainable code. This pattern provides a clear structure that facilitates application development and testing.

\textbf{Components of Model-View-Presenter}

\textbf{Model:}
The model represents the data layer and business logic of the application.
It manages access to the database, web services, or other data sources.
It may include logic to process and manipulate data before sending it to the view.

\textbf{View:}
The view is the presentation layer and is responsible for displaying the user interface.
It receives information from the presenter and updates the user interface as needed.
Ideally, the view does not contain business logic; it simply reflects the state of the model.

\textbf{Presenter:}
The presenter acts as an intermediary between the model and the view.
It retrieves data from the model and presents it to the view appropriately.
It handles user interactions and updates the model as needed.
Most of the business logic resides in the presenter.

\textbf{Characteristics of Model-View-Presenter}

\textbf{Separation of responsibilities:}
MVP facilitates clear separation of responsibilities between the model, view, and presenter layers.
This allows each component to have a specific purpose and be easier to understand and maintain.
\textbf{Testability:}
Due to the clear separation of responsibilities, code units (model, view, and presenter) are easier to test independently.
Unit tests can be performed on the presenter without the need to interact directly with the view or model.

\textbf{Code reuse:}
The modularity of the MVP pattern facilitates reuse of components in different parts of the application or even in different projects.

\textbf{Adaptability to changes:}
Because the model and view are independent of each other, it is easier to make changes to the user interface or business logic without affecting the other component.

\section{MVVM (Model View ViewModel)}

\textbf{What is the MVVM architecture pattern?}

The MVVM architecture pattern, also known as Model View ViewModel, refers to a design model that aims to achieve separation of the user interface (View) from the logic (Model). It does this with the goal that the visual aspect is completely independent.

The ViewModel resource, on the other hand, stands out as the component that will serve as a bridge between the interaction of the View and the Model.

In order to make use of the MVVM architecture pattern or mvvm model, one must understand the way in which the code of applications is factored into the appropriate classes and understand their interaction with the design components.

\textbf{Characteristics of the MVVM architecture pattern}

Among the relevant characteristics and properties of the mvvm web architecture pattern, its ability to cleanly separate the presentation of a given application and the business logic from its user interface stands out. It is worth noting that this separation between user interface and app logic helps address multiple types of development issues, facilitating testing, maintenance, and evolution processes of the system.

One of the benefits of using this architecture pattern is that it allows developers to create unit tests for the Model View and the model, without the need for the view.

Furthermore, with the mvvm architecture pattern, application design and development personnel can be able to work simultaneously and independently, each in their components during the app processes. Thus, while designers can focus on the view, developers can take care of the view components and the view model with the mvvm android architecture.
\end{document}
